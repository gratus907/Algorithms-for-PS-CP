\documentclass[landscape,8pt]{article}
\linespread{1}
\usepackage{inputenc}
\usepackage{multicol}
\setlength{\columnsep}{0.5cm}
\setlength{\columnseprule}{0.25pt}
\usepackage{amsmath}
\usepackage{graphbox}
\usepackage{amssymb}
\usepackage{amsfonts}
\usepackage{mathtools}
\usepackage{subcaption}
\usepackage{geometry}
\geometry{a4paper, landscape, margin = 0.6in}
\usepackage{enumerate}
\usepackage{amsthm}
\usepackage{physics}
\usepackage{float}
\usepackage{tabu}
\usepackage{listings}
\usepackage{wrapfig}
\usepackage{authblk}
\usepackage{verbatim}
\usepackage{graphicx}
\usepackage{xcolor}
\usepackage{fancyhdr}

\pagestyle{fancy}
\fancyhead[R]{Little Piplup}
\fancyhead[L]{Seoul National University}
\usepackage[compact]{titlesec}
\DeclarePairedDelimiter{\ceil}{\lceil}{\rceil}
\DeclarePairedDelimiter{\floor}{\lfloor}{\rfloor}
\newcommand{\st}{\text{ such that }}
\newcommand{\for}{\text{ for }}
\newcommand{\newpara}[1]{\paragraph{#1} \mbox{}\\}
\everymath{\displaystyle}
\setlength{\parskip}{0pt}
\setlength{\parsep}{0pt}
\setlength{\parindent}{0pt}
\titlespacing{\section}{0pt}{*0}{*0}
\titlespacing{\subsection}{0pt}{*1}{*0.5}
\titlespacing{\subsubsection}{0pt}{*0}{*0}
\topsep=5pt \partopsep=5pt
\title{\Huge{\textsf{Little Piplup}}}
\author{Gratus907(Wonseok Shin), Coffeetea(Jinje Han), DhDroid(Donghyun Son)}
\date{ }

\begin{document}
\maketitle
\begin{multicols}{2}\raggedcolumns
\tableofcontents
\pagebreak

\section{Settings}

  \subsection{C++}

\section{Data Structures}
  \subsection{Segment Tree}
     To deal with queries on intervals, we use segment tree.
     \verbatiminput{./code/Data_Structures/Segtree_Make.cpp}
     Updating segment Tree
     \verbatiminput{./code/Data_Structures/Segtree_Update.cpp}
     Answering queries via segment tree
     \verbatiminput{./code/Data_Structures/Segtree_Answerquery.cpp}
  \subsection{Fenwick Tree}
  \subsection{Disjoint Set Union}

\columnbreak
  \section{Mathematics}
  \subsection{Useful Mathematical Formula}
    \begin{itemize}
      \item Catalan Number : Number of valid parantheses strings with $n$ pairs
      \[
      C_n = \frac{1}{n+1}\binom{2n}{n}
      \]
    \end{itemize}
  \subsection{Number of Integer Partition}
    \verbatiminput{./code/Mathematics/Num_of_Integer_Partition.py}

  \subsection{Extended Euclidean Algorithm}
    \verbatiminput{./code/Mathematics/Extended_Euclidean_Algorithm.cpp}
  \subsection{Fast Modulo Exponentiation}
  Calculating \texttt{$x^y$ mod $p$} in $\order{\log y}$ time.
    \verbatiminput{./code/Mathematics/Fast_Modulo_Exponentiation.cpp}

  \subsection{Miller-Rabin Primality Testing}
  Base values of $a$ chosen so that results are tested to be correct up to $10^14$.
    \verbatiminput{./code/Mathematics/Miller_Rabin_Test.cpp}

  \subsection{Pollard-Rho Factorization}
    \verbatiminput{./code/Mathematics/Pollard_Rho_Algorithm.cpp}

  \subsection{Euler Totient}
  Calculating number of integers below $n$ which is coprime with $n$.
    \verbatiminput{./code/Mathematics/Euler_Phi_Function.cpp}

  \subsection{Modular Multiplicative Inverse}
    \verbatiminput{./code/Mathematics/Modular_Inverse.cpp}

\columnbreak
\section{Geometry}
  \subsection{Closest Pair Problem}

  \subsection{Smallest Enclosing Circle}

  \subsection{Convex Hull}

  \subsection{Intersection of Line Segment}

\columnbreak
\section{Graphs}
  \subsection{Topological Sorting}
  Topological sorting with dfs
    \verbatiminput{./code/Graph/Topological_Sort.cpp}

  \subsection{Dijkstra}
  $\order{E \log V}$ Single-Start-Shortest-Path.\\
  Not working for graph with minus weight.
    \verbatiminput{./code/Graph/Dijkstra_Algorithm.cpp}

  \subsection{Bellman Ford}
  $\order{EV}$ Single-Start-Shortest-Path.\\
  Not working for graph with minus cycle $\rightarrow$ must detect.
    \verbatiminput{./code/Graph/Bellman_Ford_Algorithm.cpp}

  \subsection{Floyd-Warshall}
  Works on adjacency matrix, in $\order{V^3}$.
  \verbatiminput{./code/Graph/Floyd_Warshall_Algorithm.cpp}

  \subsection{Bipartite Checking}
  \verbatiminput{./code/Graph/Bipartite_Checking.cpp}
  \subsection{MST Kruskal Algorithm}

  \subsection{MST Prim Algorithm}

  \subsection{Strongly Connected Component}

  \subsection{Ford-Fulkerson Algorithm}
\columnbreak

\section{Dynamic}
  \subsection{Largest Sum Subarray}
  Computes sum of largest sum subarray in $\order{N}$
  \verbatiminput{./code/Dynamic_Programming/Largest_Maximum_Subarray.cpp}

  \subsection{Knapsack}

  \subsection{Longest Common Subsequence}
    \verbatiminput{./code/Dynamic_Programming/Longest_Common_Substring.cpp}
  \subsection{Edit Distance}

\section{String}
  \subsection{KMP Algorithm}

  \subsection{Manacher's Algorithm}

  \subsection{Trie}

  \subsection{Rabin-Karp Hashing}

  \subsection{Aho-Corasick Algorithm}
\columnbreak
\section{Miscellaneous}
  \subsection{Useful Bitwise Functions in C++}
  \begin{verbatim}
    int __builtin_clz(int x);// number of leading zero
    int __builtin_ctz(int x);// number of trailing zero
    int __builtin_clzll(ll x);// number of leading zero
    int __builtin_ctzll(ll x);// number of trailing zero
    int __builtin_popcount(int x);// number of 1-bits in x
    int __builtin_popcountll(ll x);// number of 1-bits in x

    lsb(n): (n & -n); // last bit (smallest)
    floor(log2(n)): 31 - __builtin_clz(n | 1);
    floor(log2(n)): 63 - __builtin_clzll(n | 1);

    // compute next perm. ex) 00111, 01011, 01101, 01110, 10011, 10101..
    ll next_perm(ll v)
    {
      ll t = v | (v-1);
      return (t + 1) | (((~t & -~t) - 1) >> (__builtin_ctz(v) + 1));
    }
  \end{verbatim}
  \columnbreak
  \subsection{List of Useful Numbers}
  \begin{verbatim}

    < 10^k    prime   # of prime          < 10^k            prime
    -------------------------------------------------------------
    1             7            4          10           9999999967
    2            97           25          11          99999999977
    3           997          168          12         999999999989
    4          9973         1229          13        9999999999971
    5         99991         9592          14       99999999999973
    6        999983        78498          15      999999999999989
    7       9999991       664579          16     9999999999999937
    8      99999989      5761455          17    99999999999999997
    9     999999937     50847534          18   999999999999999989
  \end{verbatim}
\columnbreak
\section{Debugging Checkpoints}
\end{multicols}
\end{document}
